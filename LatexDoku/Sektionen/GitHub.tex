\section{GitHub}

"Build software better, together." (Motto von GitHub)\\

Für unsere Projektarbeiten verwendeten wir GitHub. GitHub ist ein webbasierter Hosting-Dienst für Softwareprojekte. Damit Emanuel und ich also bequem von zuhause aus zusammen arbeiten konnten haben wir uns ein Repository in GitHub eingerichtet. Zusammen mit dem Eclipse-Plugin EGit konnten wir unsere Arbeit austauschen und vergleichen. Obwohl es Anfangs Probleme mit der Bedienung und den verschiedenen Funktionen GitHubs und EGits gab, hat es uns doch sehr geholfen und vieles vereinfacht. So konnten wir zum Beispiel gleichzeitig an verschieden Problemen arbeiten, indem wir verschiedene Branches ( "Pfade" also Ableger des Projekts ) erstellt haben und dann an diesen Branches gearbeitet haben. Sobald dann ein Problem behoben war hat man den Pfad wieder dem Hauptprojekt hinzugefügt und konnte besprechen was genau gemacht wurde und was vielleicht noch verbessert werden muss.\\

GitHub war uns im allgemeinen eine sehr große Hilfe, aber zu Anfang auch eine große Hürde. Bis wir zurechtkamen mit den Branches, Commits ect. hat es eine Weile gedauert. Man kann aber durchaus behaupten das sich der Aufwand gelohnt hat. Ich würde jedem der ein Softwareprojekt entwickelt empfehlen mit GitHub zu arbeiten, auch wenn er alleine daran arbeitet. Denn EGit zwingt einen dazu seine Änderungen zu dokumentieren und zwischenzuspeichern. Dies ist zwar etwas nervig, aber man kann sein Projekt immer wieder auf einen beliebigen Standpunkt zurücksetzten wenn etwas komplett schiefgelaufen ist. Dies hat mir oft sehr viel Arbeit erspart. 