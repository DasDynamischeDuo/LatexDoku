\section{Einleitung}

\subsection{Vorstellung des Prokeltes}

Unser Projekt das Digital-Input Computer-Keyboard, ist wie der Name schon vermuten lässt ein 
Digitales Keyboard. Man kann damit mithilfe seines PCs und einer Tastatur Keyboard spielen. Es kann 
allerdings auch noch mehr. Der Benutzer hat die Möglichkeit verschiedene Samples ( Tonspuren die 
Instrumente darstellen ) einzustellen. Somit kann er theoretisch jedes Musikinstrument über seine 
Tastatur spielen. Außerdem kann der Benutzer gespieltes aufnehmen und abspielen lassen. Somit kann 
er mit verschiedenen Instrumenten und Tonspuren eigene Lieder zusammensampeln. 

Dieses Projekt hat Emanuel und mir sehr gut gefallen, da wir beide ein Musikinstrument spielen und 
auch allgemein Musik begeistert sind. Ein eigenes kleines Sample-Programm zu gestalten lag also 
nahe.


\subsection{Das Projektmanagement}
\subsubsection{SCRUMB}
\subsubsection{GitHub}
"Build software better, together." (Motto von GitHub)\\

Für unsere Projektarbeiten verwendeten wir GitHub. GitHub ist ein webbasierter Hosting-Dienst für 
Softwareprojekte. Damit Emanuel und ich also bequem von zuhause aus zusammen arbeiten konnten haben 
wir uns ein Repository in GitHub eingerichtet. Zusammen mit dem Eclipse-Plugin EGit konnten wir 
unsere Arbeit austauschen und vergleichen. Obwohl es Anfangs Probleme mit der Bedienung und den 
verschiedenen Funktionen GitHubs und EGits gab, hat es uns doch sehr geholfen und vieles 
vereinfacht. So konnten wir zum Beispiel gleichzeitig an verschieden Problemen arbeiten, indem wir 
verschiedene Branches ( "Pfade" also Ableger des Projekts ) erstellt haben und dann an diesen 
Branches gearbeitet haben. Sobald dann ein Problem behoben war hat man den Pfad wieder dem 
Hauptprojekt hinzugefügt und konnte besprechen was genau gemacht wurde und was vielleicht noch 
verbessert werden muss.\\

GitHub war uns im allgemeinen eine sehr große Hilfe, aber zu Anfang auch eine große Hürde. Bis wir 
zurechtkamen mit den Branches, Commits ect. hat es eine Weile gedauert. Man kann aber durchaus 
behaupten das sich der Aufwand gelohnt hat. Ich würde jedem der ein Softwareprojekt entwickelt 
empfehlen mit GitHub zu arbeiten, auch wenn er alleine daran arbeitet. Denn EGit zwingt einen dazu 
seine Änderungen zu dokumentieren und zwischenzuspeichern. Dies ist zwar etwas nervig, aber man 
kann 
sein Projekt immer wieder auf einen beliebigen Standpunkt zurücksetzten wenn etwas komplett 
schiefgelaufen ist. Dies hat mir oft sehr viel Arbeit erspart.\\

Gerade jetzt für diese Dokumentation verwenden wir auch GitHub for Windows. Ein Programm für 
Windows 
welches einem erlaubt jede Art von Datei über GitHub zu veröffentlichen und zu bearbeiten. GitHub 
hat uns also sehr viel Arbeit erspart und es uns ermöglicht obwohl wir weit auseinander wohnen ein 
Projekt auch zusammen zu erarbeiten.\\


\subsubsection{Unser Team}